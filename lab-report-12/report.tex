\input{/home/nick/latex-preambles/xelatex.tex}

\newcommand{\imagesPath}{.}

\title{
	\textbf{Δίκτυα Υπολογιστών} \\~\\
	Εργαστηριακή Άσκηση 12 \\ 
	Ασφάλεια
}
\author{}
\date{}

\begin{document}
	\maketitle
	
	\begin{tabular}{|l|l|}
		\hline
		\textbf{Ονοματεπώνυμο:} Νικόλαος Παγώνας, el18175 & \textbf{Ομάδα:} 4 (Τρίτη εξ' αποστάσεως) \\
		\hline
		\textbf{Όνομα PC/ΛΣ:} nick-ubuntu/Ubuntu 20.04.3 LTS & \textbf{Ημερομηνία:} Τρίτη 18/01/2022 \\
		\hline
		\textbf{Διεύθυνση IP:} \verb|192.168.1.15| & \textbf{Διεύθυνση MAC:} \verb|3c:2c:30:e1:1c:55| \\
		\hline
	\end{tabular}

	\section*{1. Πιστοποίηση αυθεντικότητας στο πρωτόκολλο HTTP}
		
		\subsection*{1.1}
			Ο αριθμητικός κωδικός κατάστασης είναι 401 και η φράση είναι "Authorization Required".
	
		\subsection*{1.2}
			Η επικεφαλίδα του δεύτερου μηνύματος περιέχει το επιπλέον πεδίο \verb|Authorization|.
	
		\subsection*{1.3}
			Είναι Authorization: \verb|Basic ZWR1LWR5OnBhc3N3b3Jk|.
	
		\subsection*{1.4}
			Το αποτέλεσμα της αποκωδικοποίησης είναι \verb|edu-dy:password|.
	
		\subsection*{1.5}
			Ο μηχανισμός πιστοποίησης αυθεντικότητας που παρέχει το HTTP δεν παρέχει εμπιστευτικότητα, αφού δεν έχουμε κανενός είδους κρυπτογράφηση.
	
	\section*{2. Υπηρεσία SSH - Secure SHell}
		\textbf{Σημείωση:} Επειδή για κάποιον λόγο το ssh από το μηχάνημά μας αποτύγχανε, συνδεθήκαμε στο VPN του ΕΜΠ για την εκτέλεση του ερωτήματος 2.
	
		\subsection*{2.1}
			Το SSH χρησιμοποιεί TCP.
	
		\subsection*{2.2}
			Χρησιμοποιούνται οι θύρες 22 και 46636.
	
		\subsection*{2.3}
			Η θύρα που αντιστοιχεί στο SSH είναι η 22.
	
		\subsection*{2.4}
			Η σύνταξη του φίλτρου είναι \verb|ssh|.
	
		\subsection*{2.5}
			Η έκδοση πρωτοκόλλου SSH που χρησιμοποιεί ο εξυπηρετητής είναι η 2.0, ενώ η έκδοση του λογισμικού που χρησιμοποιεί είναι \verb|OpenSSH_6.6.1_hpn13v11|. Επίσης περιλαμβάνεται το σχόλιο  \verb|FreeBSD-20140420|.
	
		\subsection*{2.6}
			Η έκδοση πρωτοκόλλου SSH που χρησιμοποιεί ο πελάτης είναι η 2.0, ενώ η έκδοση του λογισμικού που χρησιμοποιεί είναι \verb|OpenSSH_8.2p1|. Επίσης περιλαμβάνεται το σχόλιο \verb|Ubuntu-4ubuntu0.3|
	
		\subsection*{2.7}
			Υπάρχουν 10 αλγόριθμοι στη λίστα. Οι δύο πρώτοι από τους αλγορίθμους αυτής της λίστας είναι οι \verb|curve25519-sha256| και \verb|curve25519-sha256@libssh.org|.
	
		\subsection*{2.8}
			Υπάρχουν 18 αλγόριθμοι στη λίστα. Οι δύο πρώτοι από τους αλγορίθμους αυτής της λίστας είναι οι \verb|ecdsa-sha2-nistp256-cert-v01@openssh.com| και 
			\verb|ecdsa-sha2-nistp384-cert-v01@openssh.com|
	
		\subsection*{2.9}
			Οι δύο πρώτοι αλγόριθμοι είναι οι \verb|chacha20-poly1305@openssh.com| και \verb|aes128-ctr|.
	
		\subsection*{2.10}
			Οι δύο πρώτοι αλγόριθμοι είναι οι \verb|umac-64-etm@openssh.com| και \verb|umac-128-etm@openssh.com|.
	
		\subsection*{2.11}
			Οι δύο πρώτοι αλγόριθμοι είναι οι \verb|none| και \verb|zlib@openssh.com|.
			
		\subsection*{2.12}
			Ο αλγόριθμος που θα χρησιμοποιηθεί είναι ο \verb|curve25519-sha256@libssh.org|, και αυτό φαίνεται επίσης στο πεδίο "Key Exchange", σε παρένθεση \verb|(method:curve25519-sha256@libssh.org)|.
	
		\subsection*{2.13}
			Ο αλγόριθμος που θα χρησιμοποιηθεί είναι ο \verb|chacha20-poly1305@openssh.com|.
	
		\subsection*{2.14}
			Ο αλγόριθμος που θα χρησιμοποιηθεί είναι ο \verb|umac-64-etm@openssh.com|.
	
		\subsection*{2.15}
			Ο αλγόριθμος που θα χρησιμοποιηθεί είναι ο \verb|none|.
	
		\subsection*{2.16}
			Το Wireshark εμφανίζει τους επιλεχθέντες αλγορίθμους σε παρένθεση δίπλα από το πεδίο SSH Version 2.
	
		\subsection*{2.17}
			Καταγράψαμε τους τύπους:
			
			\begin{itemize}
				\item Elliptic Curve Diffie-Hellman Key Exchange Init
				\item Elliptic Curve Diffie-Hellman Key Exchange Reply
				\item New Keys
				\item Encrypted packet
			\end{itemize}
	
		\subsection*{2.18}
			Δεν μπορώ, γιατί τα πακέτα είναι κρυπτογραφημένα.
	
		\subsection*{2.19}
			To SSH είναι πιο ασφαλές από πρωτόκολλα όπως το HTTP ή το TELNET. Συγκεκριμένα, όσον αφορά την πιστοποίηση αυθεντικότητας, αυτή γίνεται με την χρήση public-private keys, η εμπιστευτικότητα επιτυγχάνεται με την κρυπτογράφηση των μηνυμάτων, ενώ η ακεραιότητα εξασφαλίζεται με το MAC.
	
		
	\section*{3. Υπηρεσία HTTPS}

		\subsection*{3.1}
			Η σύνταξη του φίλτρου σύλληψης που χρησιμοποιήσαμε είναι \verb|host bbb2.cn.ntua.gr|.

		\subsection*{3.2}
			Η σύνταξη του φίλτρου είναι \verb|tcp.flags.syn == 1 && tcp.flags.ack == 0|.

		\subsection*{3.3}
			Οι συνδέσεις έγιναν στις θύρες 80 και 443.

		\subsection*{3.4}
			Η θύρα 80 αντιστοιχεί στο HTTP και η θύρα 443 στο HTTPS.

		\subsection*{3.5}
			Στην περίπτωση HTTP έγιναν 6 συνδέσεις, ενώ στην περίπτωση HTTPS έγινε 1 σύνδεση.

		\subsection*{3.6}
			Στην περίπτωση HTTPS έχουμε ως θύρα πηγής την 51698.

		\subsection*{3.7}
			Τα τρία κοινά πεδία είναι: 
			
			\begin{itemize}
				\item Content Type: 1 byte
				\item Version: 2 bytes
				\item Length: 2 bytes
			\end{itemize}

		\subsection*{3.8}
			Οι διαφορετικές τιμές που καταγράφηκαν είναι:
		
			\begin{itemize}
				\item Change Cipher Spec - 20
				\item Alert - 21
				\item Handshake - 22
				\item Application Data - 23
			\end{itemize}

		\subsection*{3.9}
			Οι διαφορετικοί τύποι μηνυμάτων χειραψίας είναι:
		
			\begin{itemize}
				\item Client Hello - 1
				\item Server Hello - 2
				\item New Session Ticket - 4
				\item Certificate - 11
				\item Server Key Exchange - 12
				\item Server Hello Done - 14
				\item Client Key Exchange - 16
				\item Encrypted Handshake Message
			\end{itemize}

		\subsection*{3.10}
			Ο πελάτης έστειλε 1 μήνυμα Client Hello. Κάθε μήνυμα Client Hello αντιστοιχεί και σε μία σύνδεση TCP.

		\subsection*{3.11}
			Η μέγιστη έκδοση είναι η TLS 1.2.

		\subsection*{3.12}
			Το μήκος του τυχαίου αριθμού που περιέχει είναι 32 bytes. Τα πρώτα 4 είναι \verb|2e 1a 6e 1b|. Κανονικά τα 4 πρώτα bytes αναπαριστούν την χρονική στιγμή αποστολής (GMT Unix Time).

		\subsection*{3.13}
			Υπάρχουν 17 σουίτες, και οι δύο πρώτες έχουν δεκαεξαδικές τιμές: \verb|0x1301|, \verb|0x1303|.

		\subsection*{3.14}
			Θα χρησιμοποιηθεί η έκδοση TLS 1.2. Η σουίτα κωδίκων κρυπτογράφησης που τελικά επιλέχθηκε έχει όνομα \verb|TLS_ECDHE_RSA_WITH_AES_128_GCM_SHA256| και δεκαεξαδική τιμή \verb|0xc02f|.

		\subsection*{3.15}
			Το μήκος σε byte του τυχαίου αριθμού που περιέχει το μήνυμα Server Hello είναι 32. Τα πρώτα 4 bytes είναι \verb|42 63 84 18|.

		\subsection*{3.16}
			Τόσο στο Client Hello όσο και στο Server Hello, το πεδίο Compression Method έχει την τιμή \verb|null|, επομένως δεν χρησιμοποιείται συμπίεση.

		\subsection*{3.17}
			Είναι:
			
			\begin{itemize}
				\item Αλγόριθμος ανταλλαγής κλειδιών: \verb|ECDHE|
				\item Αλγόριθμος πιστοποίησης ταυτότητας: \verb|RSA|
				\item Αλγόριθμος κρυπτογράφησης: \verb|AES_128_GCM|
				\item Συνάρτηση κατακερματισμού: \verb|SHA256|
			\end{itemize}

		\subsection*{3.18}
			Με βάση το πεδίο length της επικεφαλίδας, η εγγραφή TLS που μεταφέρει το πιστοποιητικό του εξυπηρετητή έχει μήκος 4278 bytes.

		\subsection*{3.19}
			Μεταφέρονται 3 πιστοποιητικά. Τα ονόματά τους είναι: 
			
			\begin{itemize}
				\item Let's Encrypt, R3
				\item Internet Security Research Group, ISRG Root X1
				\item Digital Signature Trust Co., DST Root CA X3
			\end{itemize}

		\subsection*{3.20}
			Χρειάστηκαν 4 πλαίσια Ethernet.

		\subsection*{3.21}
			Το μήκος και των δύο δημοσίων κλειδιών που αποστέλλουν ο πελάτης και ο εξυπηρετητής αντίστοιχα είναι 32 bytes. Τα πρώτα 5 γράμματα του κλειδιού του πελάτη είναι \verb|dda63|, ενώ του εξυπηρετητή είναι \verb|fa0f9|.			

		\subsection*{3.22}
			Το μήκος της εγγραφής είναι 6 bytes συνολικά.

		\subsection*{3.23}
			Το μήκος της εγγραφής είναι 45 bytes συνολικά.

		\subsection*{3.24}
			Ναι, παρατηρήσαμε.

		\subsection*{3.25}
			Ναι, παρατηρήσαμε από την πλευρά του υπολογιστή μας.

		\subsection*{3.26}
			Το Encrypted Alert είναι υπεύθυνο για τον τερματισμό της σύνδεσης.

		\subsection*{3.27}
			Παρατηρούμε ότι η αναζήτηση επιστρέφει αποτέλεσμα μόνο στην περίπτωση του HTTP, και όχι στην περίπτωση του HTTPS.

		\subsection*{3.28}
			\begin{itemize}
				\item Πιστοποίηση αυθεντικότητας: Επιτυγχάνεται με την χρήση των certificates
				\item Εμπιστευτικότητα: Επιτυγχάνεται με την κρυπτογράφηση των μηνυμάτων
				\item Ακεραιότητα: Επιτυγχάνεται με την χρήση των hash functions
				
				Αυτό έρχεται σε αντίθεση με το HTTP, όπου δεν συμβαίνει τίποτα από τα παραπάνω.
			\end{itemize}
\end{document}