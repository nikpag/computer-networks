\input{/home/nick/latex-preambles/xelatex.tex}

\newcommand{\imagesPath}{.}

\title{
	\textbf{Δίκτυα Υπολογιστών} \\~\\
	Εργαστηριακή Άσκηση 11 \\ 
	Πλοήγηση στον παγκόσμιο ιστό
}
\author{}
\date{}

\begin{document}
	\maketitle
	
	\begin{tabular}{|l|l|}
		\hline
		\textbf{Ονοματεπώνυμο:} Νικόλαος Παγώνας, el18175 & \textbf{Ομάδα:} 4 (Τρίτη εξ' αποστάσεως) \\
		\hline
		\textbf{Όνομα PC/ΛΣ:} nick-ubuntu/Ubuntu 20.04.3 LTS & \textbf{Ημερομηνία:} Τρίτη 11/01/2022 \\
		\hline
		\textbf{Διεύθυνση IP:} \verb|192.168.1.15| & \textbf{Διεύθυνση MAC:} \verb|3c:2c:30:e1:1c:55| \\
		\hline
	\end{tabular}

	\section*{1. Ανάκτηση HTML σελίδας}
		
		\subsection*{1.1}
			Το φίλτρο απεικόνισης είναι \verb|http|.

		\subsection*{1.2}
			Ο πλοηγός ιστού μας χρησιμοποιεί το πρωτόκολλο HTTP/1.1.

		\subsection*{1.3}
			Ο εξυπηρετητής ιστού χρησιμοποιεί το πρωτόκολλο HTTP/1.1.

		\subsection*{1.4}
			Η ονομασία της μεθόδου είναι \verb|GET|.

		\subsection*{1.5}
			Δέχεται την ελληνική και την αγγλική γλώσσα.

		\subsection*{1.6}
			Ο κωδικός κατάστασης είναι 200 και η επεξηγηματική λέξη είναι "OK".

		\subsection*{1.7}
			Το σώμα της απόκρισης έχει μήκος 2648 και bytes και μεταφέρει περιεχόμενο \verb|text/html|.

		\subsection*{1.8}
			Ο τίτλος της ιστοσελίδας είναι "lab11.html" και βρίσκεται στο όνομα της καρτέλας του πλοηγού μας.

		\subsection*{1.9}
			Η σύνταξη του φίλτρου είναι:
			\begin{verbatim}
				tcp.flags.syn == 1 && tcp.flags.ack == 0 && ip.addr == 192.168.1.15|.
			\end{verbatim}

		\subsection*{1.10}
			Έγιναν 4 συνδέσεις, με θύρες πηγής 60834, 60836, 60838. 60840.

		\subsection*{1.11}
			Η σύνταξη του φίλτρου είναι \verb|http.request && ip.dst == 147.102.40.15|.

		\subsection*{1.12}
			Ο υπολογιστής μας έστειλε 8 εντολές προς τον εξυπηρετητή ιστού.

		\subsection*{1.13}
			Η σύνταξη του φίλτρου είναι \verb|http.response && ip.src == 147.102.40.15|.

		\subsection*{1.14}
			Ο πλοηγός ιστού κατέβασε 6 εικόνες.

		\subsection*{1.15}
			Ζητήθηκε να κατέβουν παράλληλα κάποιες εικόνες, αφού βλέπουμε ότι έχουν γίνει αιτήσεις για εικόνες από διαφορετικές θύρες TCP, συγκεκριμένα από τις 60826, 60834, 60836, 60838, 60840.

		\subsection*{1.16}
			Ζητήθηκε να κατέβουν ακολουθιακά οι εικόνες "favicon.ico" και "upload\_file.png", αφού και οι δύο αιτήσεις έγιναν από την θύρα 60838.
			
		\subsection*{1.17}
			Το όνομα του αρχείου που περιέχει την εικόνα αυτή είναι το \verb|favicon.ico|.
		
		\subsection*{1.18}
			Η εντολή είναι \verb|GET /favicon.ico HTTP/1.1|.
		
	\section*{2. Επανάκτηση μη-τροποποιημένης HTML σελίδας}
		
		\subsection*{2.1}
			Όχι, δεν υπάρχει.
	
		\subsection*{2.2}
			Ο κωδικός κατάστασης που επιστρέφεται είναι 200.
	
		\subsection*{2.3}
			Το αρχείο τροποποιήθηκε τελευταία φορά στις 18/12/2021, 16:01:58 GMT.
	
		\subsection*{2.4}
			Το μέγεθος του περιεχομένου που επιστρέφεται στον πλοηγό ιστού είναι 860 bytes.
	
		\subsection*{2.5}
			Το περιεχόμενο της σελίδας \verb|get2.html| είναι \verb| text/html|.
	
		\subsection*{2.6}
			Το κείμενο είναι γραμμένο σε ελληνικά και αγγλικά.
	
		\subsection*{2.7}
			Το σύνολο χαρακτήρων που χρησιμοποιείται είναι το utf-8.
	
		\subsection*{2.8}
			Παρατηρούμε ότι το παράθυρο δεδομένων του Wireshark δεν μπορεί να εμφανίσει τους ελληνικούς χαρακτήρες, γιατί χρησιμοποιεί ASCII encoding. 
	
		\subsection*{2.9}
			Ναι, υπάρχει.
	
		\subsection*{2.10}
			Αναφέρεται η ημερομηνία του ερωτήματος 2.3, δηλαδή 18/12/2021, 16:01:58 GMT.
	
		\subsection*{2.11}
			Ο κωδικός κατάστασης που επιστρέφεται είναι 304.
	
		\subsection*{2.12}
			Ο εξυπηρετητής δεν επέστρεψε ρητά τα περιεχόμενα του αρχείου get2.html γιατί υπήρχαν ήδη στην cache του πλοηγού ιστού μας, και δεν έχει μεσολαβήσει τροποποίηση τους.
	
		\subsection*{2.13}
			Ναι, υπήρξε το \verb|GET /favicon.ico HTTP/1.1|, ώστε ο πλοηγός να κατεβάσει το favicon που βρίσκεται στην καρτέλα, δίπλα στον τίτλο. 
	

	\section*{3. Επανάκτηση τροποποιημένης HTML σελίδας}

		\subsection*{3.1}
			Ο πλοηγός ιστού μας έδωσε 3 εντολές GET.

		\subsection*{3.2}
			Ο κωδικός κατάστασης που επιστρέφεται είναι 200.

		\subsection*{3.3}
			Το περιεχόμενο αυτό τροποποιήθηκε τελευταία φορά στις 26/12/2021, 20:55:00 GMT.

		\subsection*{3.4}
			To μέγεθος του περιεχομένου που επιστρέφεται είναι 601 bytes.

		\subsection*{3.5}
			Ναι, υπάρχει.

		\subsection*{3.6}
			Ο κωδικός που επιστρέφεται είναι 304.

		\subsection*{3.7}
			Ναι, υπάρχει.

		\subsection*{3.8}
			Ο κωδικός που επιστρέφεται είναι 200.

		\subsection*{3.9}
			Το περιεχόμενο τροποποιήθηκε τελευταία φορά στις 26/12/2021, 20:56:00 GMT.

		\subsection*{3.10}
			Το μέγεθος του περιεχομένου που επιστρέφεται είναι 601 bytes.

		\subsection*{3.11}
			Χρειάζεται να περιμένουμε πάνω από ένα λεπτό ώστε να αλλάξει η ώρα τελευταίας τροποποίησης και να χρειαστεί να κατεβάσουμε την καινούργια έκδοση του περιεχομένου αντί για την cached.
		
	\section*{4. Ανάκτηση εκτενούς σελίδας HTML}

		\subsection*{4.1}
			Έγινε μία σύνδεση TCP.

		\subsection*{4.2}
			Η πλευρά του υπολογιστή μας ανακοινώνει MSS = 1460 bytes, ενώ η πλευρά του server ανακοινώνει MSS = 536 bytes.

		\subsection*{4.3}
			Χρειάστηκε ένα τεμάχιο.

		\subsection*{4.4}
			HTTP/1.1 200 OK (text/html)

		\subsection*{4.5}
			Χρειάστηκαν 12 τεμάχια TCP.

		\subsection*{4.6}
			Το μήκος του περιεχομένου του αρχείο long.html είναι 7349 bytes.

		\subsection*{4.7}
			Η σύνταξη του φίλτρου είναι \verb|ip.src == 147.102.40.15|.

		\subsection*{4.8}
			Το πρώτο από τα 12 (χωρίς να μετράμε αυτά της τριπλής χειραψίας).

		\subsection*{4.9}
			Το μέγεθος του περιεχομένου καθενός από τα τεμάχια αυτά είναι 524 bytes.

		\subsection*{4.10}
			Αυτό συμβαίνει επειδή το MSS είναι σταθερό, άρα και τα segments θα είναι σταθερού μεγέθους (και επίσης επειδή οι επικεφαλίδες Ethernet, IP, TCP είναι σταθερού μεγέθους).

		\subsection*{4.11}
			Το μέγεθος του τελευταίου εξ αυτών προκύπτει ως το άθροισμα όλων των επικεφαλίδων των ανώτερων στρωμάτων + όσα δεδομένα έμειναν να μεταφερθούν. 
		
	\section*{5. Ανάκτηση HTML σελίδας με ενσωματωμένα αντικείμενα}

		\subsection*{5.1}
			Το Wireshark έχει καταγράψει 4 εντολές τύπου GET.

		\subsection*{5.2}
			Έγιναν 2 συνδέσεις TCP.

		\subsection*{5.3}
			Τα αρχεία εικόνων που ζήτησε ο πλοηγός είναι:
			
			\begin{itemize}
				\item /img/logo.gif
				\item /common\_images/pyrforos.gif
				\item /favicon.ico
			\end{itemize}

		\subsection*{5.4}
			Είναι old.ntua.gr και www.mit.edu

		\subsection*{5.5}
			\begin{itemize}
				\item GET /links.html HTTP/1.1 $\rightarrow$ \verb|147.102.40.15|
				\item GET /img/logo.gif HTTP/1.1 $\rightarrow$ \verb|2a02:26f0:c000:1ad::255e|
				\item GET /common\_images/pyrforos.gif HTTP/1.1 $\rightarrow$ \verb|2001:648:2000:de::213|
				\item GET /favicon.ico HTTP/1.1 $\rightarrow$ \verb|147.102.40.15|
			\end{itemize}
		
			Οι εικόνες logo.gif και pyrforos.gif βρίσκονται στους εξυπηρετητές old.ntua.gr και www.mit.edu, γι' αυτό και η διεύθυνση είναι διαφορετική από την 147.102.40.15.
\end{document}