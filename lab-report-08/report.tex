\input{/home/nick/latex-preambles/xelatex.tex}

\newcommand{\imagesPath}{.}

\title{
	\textbf{Δίκτυα Υπολογιστών} \\~\\
	Εργαστηριακή Άσκηση 8 \\ 
	TELNET, FTP και TFTP
}
\author{}
\date{}

\begin{document}
	\maketitle
	
	\begin{tabular}{|l|l|}
		\hline
		\textbf{Ονοματεπώνυμο:} Νικόλαος Παγώνας, el18175 & \textbf{Ομάδα:} 4 (Τρίτη εξ' αποστάσεως) \\
		\hline
		\textbf{Όνομα PC/ΛΣ:} nick-ubuntu/Ubuntu 20.04.3 LTS & \textbf{Ημερομηνία:} Τρίτη 07/11/2021 \\
		\hline
		\textbf{Διεύθυνση IP:} \verb|192.168.1.15| & \textbf{Διεύθυνση MAC:} \verb|3c:2c:30:e1:1c:55| \\
		\hline
	\end{tabular}

	\section*{1. TELNET}

		\subsection*{1.1}
			Το TELNET χρησιμοποιεί το πρωτόκολλο TCP.
		
		\subsection*{1.2}
			Οι θύρες του πρωτοκόλλου μεταφοράς που χρησιμοποιούνται για την επικοινωνία είναι οι 23 και 38212.
		
		\subsection*{1.3}
			Στο πρωτόκολλο TELNET αντιστοιχεί η θύρα 23.
		
		\subsection*{1.4}
			Η σύνταξη του φίλτρου απεικόνισης είναι \verb|telnet|.
		
		\subsection*{1.5}
			\begin{itemize}
				\item 147.102.40.15 $\rightarrow$ 192.168.1.15: Do Echo 
				\item 192.168.1.15 $\rightarrow$ 147.102.40.15: Won't Echo
				\item 147.102.40.15 $\rightarrow$ 192.168.1.15: Will Echo
				\item 192.168.1.15 $\rightarrow$ 147.102.40.15: Do Echo
			\end{itemize}
		
		\subsection*{1.6}
			Ναι, ο edu-dy.cn.ntua.gr ζητάει από τον υπολογιστή μας να επαναλαμβάνει τους χαρακτήρες που λαμβάνει (Do Echo), αλλά ο υπολογιστής μας δεν δέχεται (Won't Echo). 
		
		\subsection*{1.7}
			Όχι, ο edu-dy.cn.ntua.gr δεν ζητάει από τον υπολογιστή μας να \underline{μην} επαναλαμβάνει τους χαρακτήρες που λαμβάνει.
		
		\subsection*{1.8}
			Nαι, ο edu-dy.cn.ntua.gr προτίθεται να επαναλαμβάνει τους χαρακτήρες που λαμβάνει (Will Echo).
		
		\subsection*{1.9}
			Ναι, έχει προηγηθεί (Do Echo).
		
		\subsection*{1.10}
			Κατά τη μεταφορά του ονόματος χρήστη, βλέπουμε ότι για κάθε χαρακτήρα που πληκτρολογούμε (και άρα στέλνουμε στον edu-dy.cn.ntua.gr), ο edu-dy.cn.ntua.gr μας απαντά αμέσως πίσω με τον ίδιο χαρακτήρα. 
			
		\subsection*{1.11}
			Το παραπάνω γίνεται διότι ο edu-dy.cn.ntua.gr έχει κάνει Will Echo και ο υπολογιστής μας έχει κάνει Do Echo, δηλαδή έχει γίνει η συμφωνία ότι ο edu-dy.cn.ntua.gr θα επαναλαμβάνει τους χαρακτήρες που λαμβάνει.
		
		\subsection*{1.12}
			Η σύνταξη του φίλτρου είναι: \\
			\verb|ip.version == 4 && telnet && ip.src == 192.168.1.15 && ip.dst == 147.102.40.15|.
		
		\subsection*{1.13}
			Χρειάζονται 5 πακέτα, ένα για κάθε χαρακτήρα του "abcd" και ένα για το <Enter> (\textbackslash r)
		
		\subsection*{1.14}
			Και πάλι χρειάζονται 5 πακέτα, ένα για κάθε χαρακτήρα του "efgh" και ένα για το <Enter> (\textbackslash r)
		
		\subsection*{1.15}
			Όχι, δεν την στέλνει.
		
		\subsection*{1.16}
			Όχι, δεν την παρατηρήσαμε.
		
		\subsection*{1.17}
			Ο κωδικός δεν εμφανίζεται για λόγους ασφαλείας.
		
		\subsection*{1.18}
			Η υπηρεσία Telnet δεν είναι ασφαλής, αφού η πληροφορία στέλνεται χωρίς καμία κρυπτογράφηση, και όποιος παρακολουθεί την κίνηση πακέτων μεταξύ των δύο πλευρών μπορεί να υποκλέψει προσωπικά δεδομένα, όπως κάναμε μόλις με το username μέσω του Wireshark.
		
	\section*{2. FTP}
		
		\subsection*{2.1}
			Το φίλτρο σύλληψης που χρησιμοποιήσαμε είναι \verb|host edu-dy.cn.ntua.gr|
		
		\subsection*{2.2}
			Το option \verb|-d| σημαίνει ότι έχουμε κάνει enable το debugging.
		
		\subsection*{2.3}
			Το FTP χρησιμοποιεί το TCP σαν πρωτόκολλο μεταφοράς.
		
		\subsection*{2.4}
			\begin{itemize}
				\item Εντολές ελέγχου: Θύρες 21 και 53396.
				\item Εντολές μεταφοράς δεδομένων: Θύρες 20 και 47745.
			\end{itemize}
		
		\subsection*{2.5}
			Η σύνδεση TCP για τη μεταφορά δεδομένων FTP γίνεται από την πλευρά του εξυπηρετητή.
		
		\subsection*{2.6}
			Ο πελάτης έστειλε τις εντολές:
		
			\begin{itemize}
				\item USER anonymous
				\item PASS labuser@cn
				\item SYST
				\item HELP
				\item PORT 147, 102, 131, 25, 186, 129
				\item LIST
				\item QUIT
			\end{itemize}
								
		\subsection*{2.7}
			Οι εντολές αυτές εμφανίζονται αυτούσιες στο παράθυρο εντολών ως εξής:
			
			\begin{itemize}
				\item - - - > USER anonymous
				\item - - - > PASS XXXX (εδώ δεν εμφανίζεται το labuser@cn)
				\item - - - > SYST
				\item - - - > HELP
				\item - - - > PORT 147, 102, 131, 25, 186, 129
				\item - - - > LIST
				\item - - - > QUIT
			\end{itemize}
		
		
		\subsection*{2.8}
			Το όνομα χρήστη μεταφέρεται με την εντολή USER.
		
		\subsection*{2.9}
			Για να μεταφερθεί το όνομα χρήστη χρειάζεται ένα πακέτο.
		
		\subsection*{2.10}
			Ο κωδικός χρήστη μεταφέρεται με την εντολή PASS.
		
		\subsection*{2.11}
			Για να μεταφερθεί ο κωδικός χρήστη χρειάζεται ένα πακέτο.
		
		\subsection*{2.12}
			Παρατηρούμε ότι το όνομα και ο κωδικός χρήστη μεταφέρονται χωρίς κρυπτογράφηση και στο FTP. Αντίθετα με το TELNET όμως, η πληροφορία στο FTP δεν μεταφέρεται χαρακτήρα-χαρακτήρα, αλλά το όνομα χρήστη μεταφέρεται σαν ολόκληρη συμβολοσειρά σε ένα μόνο πακέτο (το ίδιο ισχύει και για τον κωδικό χρήστη).
		
		\subsection*{2.13}
			Όχι, η εντολή \verb|help| δεν μεταφράζεται σε εντολή του πρωτοκόλλου FTP.
		
		\subsection*{2.14}
			Δύο εντολές που δεν υποστηρίζονται από τον εξυπηρετητή είναι η \verb|PBSZ| και η \verb|PROT|. Αυτό φαίνεται επειδή είναι επισημασμένες με αστεράκι.
		
		\subsection*{2.15}
			Από τον υπολογιστή μας στάλθηκε 1 πακέτο, ενώ από τον εξυπηρετητή στάλθηκαν 9 πακέτα.
		
		\subsection*{2.16}
			Ο εξυπηρετητής δηλώνει ότι τελείωσε η αποστολή με το να βάλει στην τελευταία γραμμή τον κωδικό απάντησης ακολουθούμενο από ένα κενό διάστημα (και όχι παύλα).
		
		\subsection*{2.17}
			Οι πρώτοι 4 δεκαδικοί αριθμοί παριστάνουν την IPv4 διεύθυνση του υπολογιστή μας.
		
		\subsection*{2.18}
			Η θύρα αυτή προκύπτει ως εξής:
			
			\[
				\text{Θύρα} = \text{5ος δεκαδικός αριθμός εντολής PORT} * 256 + \text{6ος δεκαδικός αριθμός εντολής PORT}.
			\]
			
			Επιβεβαιώνουμε ότι με αυτόν τον τρόπο η θύρα που προκύπτει είναι αυτή που βρήκαμε προηγουμένως στο ερώτημα 2.4, δηλαδή η 47745.
		
		\subsection*{2.19}
			Η εντολή LIST.
		
		\subsection*{2.20}
			Η εντολή PORT προηγείται της LIST διότι πρόκειται να γίνει νέα τριπλή χειραψία για την μετάδοση των δεδομένων.
		
		\subsection*{2.21}
			Η εντολή bye μεταφράζεται στην εντολή QUIT.
		
		\subsection*{2.22}
			Ο εξυπηρετητής FTP ανταποκρίνεται με το μήνυμα "221 Goodbye."
		
		\subsection*{2.23}
			Η σύνταξη του φίλτρου είναι \verb|tcp.flags.fin == 1|.
		
		\subsection*{2.24}
			Η απόλυση των συνδέσεων ελέγχου και δεδομένων έγινε από την πλευρά του πελάτη.
		
		\subsection*{2.25}
			\begin{itemize}
				\item Εντολές ελέγχου: Θύρες 21, 53974
				\item Εντολές μεταφοράς δεδομένων: Θύρες 33832, 47957
			\end{itemize}
		
		\subsection*{2.26}
			Οι εντολές είναι οι εξής:
			
			\begin{itemize}
				\item FEAT
				\item USER anonymous
				\item PASS gvfsd-ftp-1.44.1@example.com
				\item TYPE I
				\item OPTS UTF8-ON
				\item SYST
				\item SITE HELP
				\item PWD
				\item CWD /
				\item PASV
				\item LIST -a
			\end{itemize}
		
		\subsection*{2.27}
			Στην δική μας περίπτωση χρησιμοποιήθηκε ως όνομα χρήστη το \verb|anonymous|, ενώ ως κωδικός χρήστη το \verb|gvfsd-ftp-1.44.1@example.com|.
		
		\subsection*{2.28}
			Για την εμφάνιση της λίστας αρχείων χρησιμοποιήθηκε η εντολή \verb|LIST -a|.
		
		\subsection*{2.29}
			Ο εξυπηρετητής ανταποκρίνεται με το μήνυμα: \\
			\verb|227 Entering Passive Mode (147,102,40,15,187,85)|.
		
		\subsection*{2.30}
			Η εγκατάσταση της σύνδεσης TCP γίνεται από την πλευρά του υπολογιστή μου.
		
		\subsection*{2.31}
			Η θύρα του εξυπηρετητή που χρησιμοποιείται για τη μεταφορά δεδομένων FTP είναι η 47957. Αυτός ο αριθμός μπορεί να προκύψει και μέσω της εντολής PORT ως εξής:
			
			\[
			\text{Θύρα} = \text{5ος δεκαδικός αριθμός εντολής PORT} * 256 + \text{6ος δεκαδικός αριθμός εντολής PORT}.
			\]
		
		\subsection*{2.32}
			Η θύρα από την πλευρά του πελάτη είναι η πρώτη διαθέσιμη.
		
		\subsection*{2.33}
			Στάλθηκαν 4 μηνύματα από τον εξυπηρετητή, με μέγεθος δεδομένων 524, 524, 524 και 155 bytes αντίστοιχα.
		
		\subsection*{2.34}
			Τα περιεχόμενα του καταλόγου έχουν θρυμματιστεί, γι' αυτό έχουμε 3 πακέτα μήκους ακριβώς 576 bytes και ένα 155. 
		
		\subsection*{2.35}
			Η απόλυση της σύνδεσης που αφορά τις εντολές ελέγχου γίνεται από τον εξυπηρετητή.
		
		\subsection*{2.36}
			Η απόλυση της σύνδεσης που αφορά τα μηνύματα δεδομένων γίνεται από τον πελάτη.
		
	\section*{3. TFTP}
		
		\subsection*{3.1}
			To TFTP χρησιμοποιεί το πρωτόκολλο μεταφοράς UDP.
		
		\subsection*{3.2}
			\begin{itemize}
				\item Θύρα πηγής: 34175
				\item Θύρα προορισμού: 69 
			\end{itemize}
		
		\subsection*{3.3}
			\begin{itemize}
				\item Θύρα πηγής (εξυπηρετητής): 16799
				\item Θύρα προορισμού (πελάτης): 34175
			\end{itemize}
		
		\subsection*{3.4}
			Στο TFTP αντιστοιχεί η θύρα 69.
		
		\subsection*{3.5}
			Οι αριθμοί θυρών που χρησιμοποιούνται κατά την μεταφορά δεδομένων επιλέγονται τυχαία.
		
		\subsection*{3.6}
			Η μεταφορά του αρχείου rfc1350.txt γίνεται με ASCII.
		
		\subsection*{3.7}
			Ο τρόπος μεταφοράς καθορίζεται από το πρώτο μήνυμα που στέλνει ο πελάτης, μέσω του πεδίου Type του TFTP, το οποίο έχει την τιμή \verb|netascii|. 
		
		\subsection*{3.8}
			Οι τύποι μηνυμάτων που παρατηρήσαμε είναι: 
			\begin{itemize}
				\item Read Request
				\item Data Packet
				\item Acknowledgment
			\end{itemize}
		
		\subsection*{3.9}
			Στο TFTP ο πελάτης στέλνει μηνύματα επιβεβαίωσης (Acknowledgment) όπου αναγράφεται το block δεδομένων που λήφθηκε με επιτυχία.

		\subsection*{3.10}
			Γι' αυτόν τον σκοπό χρησιμοποιείται ο τύπος μηνυμάτων Acknowledgment, του πεδίου Opcode.

		\subsection*{3.11}
			Το μέγεθος των μηνυμάτων TFTP (πλην του τελευταίου) είναι 516 bytes.

		\subsection*{3.12}
			Το μέγεθος δεδομένων είναι 512 bytes.

		\subsection*{3.13}
			Ο πελάτης αντιλαμβάνεται το τέλος της μετάδοσης δεδομένων αν λάβει ένα μήνυμα με μέγεθος δεδομένων μικρότερο από 512 bytes.
\end{document}